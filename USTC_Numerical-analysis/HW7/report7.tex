\documentclass{article}
\textheight 23.5cm \textwidth 15.8cm
%\leftskip -1cm
\topmargin -1.5cm \oddsidemargin 0.3cm \evensidemargin -0.3cm
%\documentclass[final]{siamltex}

\usepackage{verbatim}
\usepackage{fancyhdr}
\usepackage{amssymb,ctex}
\usepackage{mathrsfs}
\usepackage{latexsym,amsmath,amssymb,amsfonts,epsfig,graphicx,cite,psfrag}
\usepackage{eepic,color,colordvi,amscd}
\usepackage{enumerate}
\usepackage{booktabs}
\usepackage{graphicx}
\usepackage{float}
\usepackage{multirow}


\title{Numerical Analysis Homework7}
\author{Zhang Jiyao,PB20000204}

\begin{document}
	\maketitle
	
	\section{Introduction}
	编写一个子程序,用来执行定义在任意区间$[a,b]$上的函数$f$的龙贝格算法。用户要具体指定阵列中所计算的行数,并且当计算完成后要看到整个阵列。编写一个主程序并且用下列积分测试这个子程序。
	
	\begin{equation}
		\int_0^{1}\frac{sinx}{x}dx
	\end{equation}

	\begin{equation}
	\int_{-1}^{1}\frac{cosx-e^x}{sinx}dx
    \end{equation}

	\begin{equation}
	\int_1^{\infty}(xe^x)^{-1}dx
    \end{equation}
		
	\section{Method}
	
	考虑龙贝格算法:对于任意区间$[a,b]$,设$h_0=b-a$,$h_n=\frac{h_{n-1}}{2} $  $(n\geq 1)$
	
	使用上述公式,我们有
	$$ R(0,0)=\frac{1}{2}(b-a)[f(a)+f(b)]$$
	$$ R(n,0)=R(n-1,0)+h_n\sum_{i=1}^{2^{n-1}}f(a+(2i-1)h_n)$$
	
	则对于一个适度的$M$值,计算出$R(0,0),R(1,0),R(2,0),...,R(M,0)$的值。然后不需要依赖被积函数$f$的值,根据公式
	
	$$R(n,m)=R(n,m-1)+\frac{1}{4^m-1}[R(n,m-1)-R(n-1,m-1)]$$
	
	可以求得全部的阵列。
	
	编写对应的子程序Romberg.m。用户在调用时需要自行输出对应的区间,被积函数,以及输出阵列的行数$M$。
	
	注意对于本题的三个积分,我们不能直接代入区间计算。因为它们是反常积分。我们可以取一个很小的$\epsilon$,例如对第一个积分,在区间$[\epsilon,1]$上进行积分。
	
	对于第一个积分,注意到$\lim\limits_{x\to 0^{+}}\frac{sinx}{x}=1$,因此取充分小的$\epsilon$,在区间$[0,\epsilon]$上积分的贡献是小于$\epsilon$的,因此可以忽略不计。
	
	对于第二个积分,注意到0是一个间断点,因此在区间$[-1,-\epsilon]$以及$[\epsilon,1]$上分别来讨论即可。也是注意到极限$\lim\limits_{x\to 0}\frac{cosx-e^x}{sinx}=1$
	
	对于第三个积分,进行变量替换$x=\frac{1}{t}$,原式化为
	$$ \int_{0}^{1}\frac{1}{te^{\frac{1}{t}}}dt$$
	
    并且注意到极限$\lim\limits_{x\to 0^{+}}\frac{1}{xe^{\frac{1}{x}}}=0$方法同第一个。
	
	
	\section{Results}
	
	求得的结果如下图所示。依照次序,分别是三个积分求得的值。
	
	\begin{table}[H]
		\centering
		\begin{tabular}{|l|l|l|l|l|l|l|}
			\hline
			0.920735 & 0 & 0 & 0 & 0 & 0 & 0 \\ \hline
			0.939793 & 0.946146 & 0 & 0 & 0 & 0 & 0 \\ \hline
			0.944514 & 0.946087 & 0.946083 & 0 & 0 & 0 & 0 \\ \hline
			0.945691 & 0.946083 & 0.946083 & 0.946083 & 0 & 0 & 0 \\ \hline
			0.945985 & 0.946083 & 0.946083 & 0.946083 & 0.946083 & 0 & 0 \\ \hline
			0.946059 & 0.946083 & 0.946083 & 0.946083 & 0.946083 & 0.946083 & 0 \\ \hline
			0.946077 & 0.946083 & 0.946083 & 0.946083 & 0.946083 & 0.946083 & 0.946083 \\ \hline
		\end{tabular}
	\end{table}
	
	\begin{table}[H]
		\centering
		\begin{tabular}{|l|l|l|l|l|l|l|}
			\hline
			-1.95171 & 0 & 0 & 0 & 0 & 0 & 0 \\ \hline
			-2.06277 & -2.09979 & 0 & 0 & 0 & 0 & 0 \\ \hline
			-2.1451 & -2.17255 & -2.1774 & 0 & 0 & 0 & 0 \\ \hline
			-2.19342 & -2.20952 & -2.21199 & -2.21254 & 0 & 0 & 0 \\ \hline
			-2.2194 & -2.22806 & -2.22929 & -2.22957 & -2.22963 & 0 & 0 \\ \hline
			-2.23284 & -2.23732 & -2.23794 & -2.23808 & -2.23811 & -2.23812 & 0 \\ \hline
			-2.23968 & -2.24196 & -2.24227 & -2.24234 & -2.24235 & -2.24236 & -2.24236 \\ \hline
		\end{tabular}
	\end{table}

\begin{table}[H]
	\centering
	\begin{tabular}{|l|l|l|l|l|l|l|}
		\hline
		0.18394 & 0 & 0 & 0 & 0 & 0 & 0 \\ \hline
		0.227305 & 0.24176 & 0 & 0 & 0 & 0 & 0 \\ \hline
		0.219834 & 0.217344 & 0.215716 & 0 & 0 & 0 & 0 \\ \hline
		0.219351 & 0.21919 & 0.219313 & 0.21937 & 0 & 0 & 0 \\ \hline
		0.219384 & 0.219394 & 0.219408 & 0.21941 & 0.21941 & 0 & 0 \\ \hline
		0.219384 & 0.219384 & 0.219383 & 0.219383 & 0.219383 & 0.219383 & 0 \\ \hline
		0.219384 & 0.219384 & 0.219383 & 0.219383 & 0.219383 & 0.219383 & 0.219383 \\ \hline
	\end{tabular}
\end{table}


	\section{Discussion}
	
	这三个积分我们都可以求得准确值。对比误差可知,Romberg积分求得的结果还是较为准确的,收敛速度也较快。
	

	
	\section{Computer Code}
	\verbatiminput{Romberg.m}
	\verbatiminput{main.m}

\end{document}