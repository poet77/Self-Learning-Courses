\documentclass{article}
\textheight 23.5cm \textwidth 15.8cm
%\leftskip -1cm
\topmargin -1.5cm \oddsidemargin 0.3cm \evensidemargin -0.3cm
%\documentclass[final]{siamltex}

\usepackage{verbatim}
\usepackage{fancyhdr}
\usepackage{amssymb,ctex}
\usepackage{mathrsfs}
\usepackage{latexsym,amsmath,amssymb,amsfonts,epsfig,graphicx,cite,psfrag}
\usepackage{eepic,color,colordvi,amscd}
\usepackage{enumerate}
\usepackage{booktabs}
\usepackage{graphicx}



\title{Numerical Analysis Homework3}
\author{Zhang Jiyao,PB20000204}

\begin{document}
	\maketitle
	
	\section{Introduction}
	
	对函数
	$$ f(x)=e^x,x\in [0.1]$$
	构造等距节点的样条插值函数,对以下两种类型的样条函数
	
	(1):一次分片线性样条
	
	(2):满足$S'(0)=1$,$S'(1)=e$的三次样条
	
	并计算如下误差
	$$max_i\{\left|f(x_{i-\frac{1}{2}})-S(x_{i-\frac{1}{2}})\right|, i=1,...,N \}$$
	
	说明:其中$x_{i-\frac{1}{2}}$是每个小区间的中点.对$N=5,10,20,40$比较以上两组节点的结果。并且计算该算法的收敛阶,公式如下
	
	$$Ord=\frac{ln(\frac{Error_{old}}{Error_{now}})}{ln(\frac{N_{now}}{N_{old}})}$$
	
	
	
	\section{Method}
	
	先讨论第一种一次分片线性样条的情况,该情况下比较简单。因为构造的一次分片线性样条在每个区间上面都是线性函数,那么只需要根据每个区间两侧端点的值,构造对应区间上的线性函数即可。为了计算一点的值,只需要利用以下公式即可
	$$y=\frac{x-x_0(i+1)}{x_0(i)-x_0(i+1)}y_0(i)+\frac{x-x_0(i)}{x_0(i+1)-x_0(i)}y_0(i+1)$$
	
	说明:$x$是待求节点,$y$是待求值,$x_0(i)$是第$i$个区间的左端点,$y_0(i)$是其对应的值
	
	接下来讨论三次样条的情况。因为三次样条本身已经确定了$4n-2$个自由度,只需要给出边界的两个自由度。而本题中$S'(0)=1$,$S'(1)=e$满足要求。记$M_i=S^{''}(x_i)$,则我们可以给出$M_i$满足的线性方程组
	
	\begin{equation}
		\left[
		\begin{array}{cccc}
			u_{1}& h_{1} &  &   \\
			h_{1}& u_{2} &h_{2}&   \\
			  &h_{2} &u_{3}&h_{3}       \\
		 	        &... & ...  &  \\
			            &h_{n-3} &u_{n-2} &h_{n-2} \\
			            &        &h_{n-2} &u_{n-1} 
			
		\end{array}
		\right ]
		\left[
		\begin{array}{cccc}
			M_{1}\\
			M_{2}\\
			M_{3}\\
			... \\
			M_{n-2}\\
			M_{n-1}
		\end{array}
		\right ]
		=
		\left[
		\begin{array}{cccc}
				v_{1}\\
			v_{2}\\
			v_{3}\\
			... \\
			v_{n-2}\\
			v_{n-1}
		\end{array}
		\right ]
	\end{equation}

然后我们采用追赶法解这个线性方程组。

其中$h_i=x_{i+1}-x_i$,$u_i=2(h_i+h_{i+1})$,$b_i=6\frac{y_{i+1}-y_i}{h_i}$,$v_i=b_i-b_{i-1}$,再结合边界条件$S'(0)=1$,$S'(1)=e$即可得出结论。

接下来讨论计算样条函数在某一确定点处的值。先确定$x$是在哪个区间中,比如说是$[x_i,x_{i+1}]$中,那么重写$S_i(x)$的表达式为
$$S_i(x)=y_i+(x-t_i)[C_i+(x-t_i)[B_i+(x-t_i)A_i]]$$

其中,$A_i=\frac{M_{i+1}-M_i}{6h_i}$,$B_i=\frac{M_i}{2}$,$C_i=-\frac{h_iM_{i+1}}{6}-\frac{h_iM_i}{3}+\frac{y_{i+1}-y_i}{h_i}$

	\section{Results}
	

\begin{table}[h]
	\setlength\tabcolsep{25pt}
	\begin{tabular}{|l|l|l|l|l|}
		\hline
		n  & Method (1) error & order  & Method (2) error & order  \\ \hline
		5  & 0.0123           & -----  & 1.0907e-05       & -----  \\ \hline
		10 & 0.0032           & 1.9288 & 6.9559e-07       & 3.9709 \\ \hline
		20 & 8.2853e-04       & 1.9642 & 4.3871e-08       & 3.9869 \\ \hline
		40 & 2.0973e-04       & 1.9820 & 2.7538e-09       & 3.9938 \\ \hline
	\end{tabular}
\end{table}
	
	\section{Discussion}
	
	观察数据可知,两种方法都有随着$N$越来越大,误差减小,收敛阶增大的趋势。
	
	首先单独讨论。第一种方法的误差相对较大,在$N=5$时误差已经达到了0.01,可见分片线性样条并不是一个优秀的插值函数。并且观察到$N$从20变到40时,误差反而增大,这说明这种算法不够稳定,可能在某些节点处会有较大的误差。
	
	相比之下,第二种方法的误差就小了很多,误差最大也不过$1e-5$量级.并且误差随着$N$的增大越来越小,较为稳定。
	
	两者之间对比的话,无论误差还是收敛阶,第二种方法都完胜第一种。综上我们可以认为三次样条插值是更好的算法。
	
	\section{Computer Code}
	
	\verbatiminput{main.m}
	\verbatiminput{sol_linear.m}
	\verbatiminput{cubicspline.m}
	\verbatiminput{Chase_method.m}

\end{document}